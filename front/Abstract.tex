% % Abstract

\thispagestyle{empty}
\pdfbookmark[0]{Abstract}{Abstract} % Bookmark name visible in a PDF viewer

\begin{center}
%	\bigskip

    {\normalsize \href{http://www.sussex.ac.uk/}{\myUni} \\} % University name in capitals
    {\normalsize \myFaculty \\} % Faculty name
    {\normalsize \myDepartment \\} % Department name
    \bigskip\vspace*{.02\textheight}
    {\Large \textsc{Doctoral Thesis}}\par
    \bigskip
    
    {\rule{\linewidth}{1pt}\\%[0.4cm]
    \Large \myTitle \par} % Thesis title
    \rule{\linewidth}{1pt}\\[0.4cm]
    
    \bigskip
	{\normalsize by \myName \par} % Author name
    \bigskip\vspace*{.06\textheight}
\end{center}

    {\centering\Huge\textsc{\textbf{Abstract}} \par}
    \bigskip



    \noindent This paper resituates multisensory augmented reality (MSAR) as an artistic medium for the creation of interactive and expressive works by computational artists. If an AR system can be thought of as one that combines real and virtual processes, is interactive in real-time, and is registered in three dimensions; why do we witness the majority of AR applications utilising primarily visual displays of information? In this paper, I propose a practice-led compositional approach for developing `MSAR Experiences', arguing that, as an medium that combines real and virtual multisensory processes, it must explored with a multisensory approach. The paper further outlines the study methods that I will use to evaluate the developed experiences. The outcome of this project is the practice-led method as well as MSAR hardware, software and experiences that are developed and evaluated.
 
