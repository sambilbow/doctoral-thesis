% ---------------------------------------------
%  _       _
% (_)_ __ | |_ _ __ ___
% | | '_ \| __| '__/ _ \
% | | | | | |_| | | (_) |
% |_|_| |_|\__|_|  \___/
% ---------------------------------------------
%*[ ]   re-draft
% ---------------------------------------------
\chapter{Introduction}
\label{sec: introduction}
\markboth{}{Introduction}
\epigraph{\emph{I dream of instruments obedient to my thought and which with their contribution of a whole new world of unsuspected sounds, will lend themselves to the exigencies of my inner rhythm}}{\citep{varese1966}}
% ---------------------------------------------
\section{Multisensory AR as a Medium for Embodied Musical Performance}\label{sec: introduction-summary}
In the last twenty years, computational technology has become more and more expressive, the systems we engage with becoming increasingly interactive. Due to this, the arts (especially forms of sound-driven digital art) have embraced new technologies in tandem with, and often contributing to their development. This has lead to more human-centred methods of designing tools of digital art creation, moving away from black-boxed (undisclosed to the audience) interactions, towards more performative and suggestive interactions, which often include the audience. Among some of the technologies that have seen nascent in the arts is XR (the X stands for nothing and everything, the R stands for Reality), a group of technologies that promise to radically rethink the way we approach reality. 

The category of XR comprises the technologies of: Virtual Reality (VR) - technology that promises to immerse us in a completely synthetic virtual world; and Augmented Reality (AR) -  technology that promises to immerse us in our own real world through the computational mediation of virtual processes. This leads to the augmentation, diminution, hybridisation, or extension of our own experience of reality, blurring and perhaps dissolving the line between our virtual and real selves. The present thesis makes no distinction between `Mixed Reality' (MR) and Augmented Reality, and adopts AR as an umbrella term for all contemporary (and mostly marketing) uses of the term MR.

AR, as mentioned, has seen sporadic use in the arts in the last twenty years. Its use in other disciplines, such as medical science, manufacturing and repair, annotation and visualisation, and the military \citep{azuma1997}, have all been typically characterised by its use as technology that aids in the \textbf{visual overlay and alignment} of \textbf{virtual graphics} (text, image, animation, video) onto our \textbf{real world environment}. Within these fields, this has a potential myriad important uses, from training junior doctors in dummy procedures by overlaying the steps of a difficult surgery, to allowing three-dimensional networked collaboration across vast distances. Within the arts, its use has also been typically limited to this visual overlay approach. Despite being only a fraction of the possible forms of AR, this view of AR dominates the market of consumer AR devices, from head-mounted, to handheld and projective technologies, the majority deal with overlaying visual information.  By taking a do-it-yourself (DIY) approach to designing and implementing custom digital musical tools through hardware and software, the present practice-based research thesis approaches AR through a multisensory lens to designing rich, expressive experience. The importance of such an approach is in its ability help redefine, and understand more about the technologies with which we are choosing to mediate our daily lives with, and also about how our own perception of digital media is affected through this embodied usage. 



% ---------------------------------------------
\section{Personal Motivations}\label{sec: introduction-motivations}
My own motivations for taking this multisensory, embodied approach to designing for AR is in the potential expressive affordance of the tool in creative rich, sensory artwork, which seeks to teach us more about the way we perceive ourselves, others, and our environment. Coming from a background in music technology, specifically in developing and evaluating digital music tools and instruments, the nature of typical `visual overlay' AR is incompatible with its use as a digital music tool, and as such, it seems not only possible, but reasonable for it to be redefined to encapsulate all of the senses. In wanting to foster embodied experience between audience members (collaborative expression), AR offers a unique way of mediating and co-constructing (with them) the space in which they embody being and knowing. Thus, my motivation is also within the advancement of our understanding this hybrid space, in which unique relationships between virtual and real processes are played out.



% ---------------------------------------------
\section{Key Research Questions}\label{sec: introduction-researchquestions}


AR has seen nascent implementation as a creative medium in the arts, and more specifically music composition and performance. Therefore, the research areas and questions for the present thesis are based in the intersection between the arts, technology, sensory perception:

\RQallsub

% ---------------------------------------------
\section{Aims \& Objectives of the Thesis}\label{sec: introduction-aims}
As well as answering and contributing understanding to the above research questions, this thesis aims to:

\begin{itemize}
    \item Conduct a thorough literature and practice review of the field of AR within the arts
    \item Contribute to the understanding of the `tools of use' that arise from ARs application within the arts
    \item Apply theories of space, materiality, and embodiment to contemporary AR design, use, and evaluation.
    \item Contribute to the practice field of AR arts by designing and evaluating multisensory AR tools for creative and collaborative expression.
    \item Contribute to the understanding of how such tools might affect our perceptions of self, others and our real, virtual, and hybrid environments. 
\end{itemize}



% ---------------------------------------------
\section{Research Methods}\label{sec: introduction-methods}
The research methods used to answer my research questions, and achieve my aims and objectives above involve taking a practice-based approach to designing multisensory digital musical instruments using AR. Involved too, will be qualitatively (coded interview), and quantitatively (survey of experience) evaluating these instruments and tools, and iteratively redesigning them to provide rich and immersive experience.



% ---------------------------------------------
\section{Outline of Chapters}\label{sec: introduction-outline}
Chapter two starts with an in-depth history of AR as an emerging technology within the field of Human Computer Interaction (HCI), examining the historical barriers to usage such as portability, price that have led towards its popularity in a variety of forms today. Among these forms, the majority deal with simple and descriptive visual additions in the form of text, image, animation or video overlay. I make a case for `multisensory' AR, i.e. the designing of content or experience for multiple senses to counter not only this bias toward visual design, but also as a method of attaining richer and more novel experiences of AR. This case is made by examining current advancements in multisensory HCI applications and enabling technologies. In the second part of the chapter, I posit through a review of contemporary art practice, the usefulness of AR use within the arts in contributing to the designing and evaluation of these tools, enabling new aesthetic and multisensory experiences, collaborative expression, and agency.

In chapter three, I identify three lenses through which to view, not only the potential of modern-day applications of multisensory AR art, but also the potential of the medium itself. The first is through considering the intervention of AR processes in real and virtual spaces, how this effects the construction of these spaces, the implications this in-turn has for the type of art that is created, and the privacy and security of people co-located in these spaces. The second is through a consideration of the materiality and form of AR processes for both designer, user, and audience; not only by focusing on the tangibility and aesthetics of intervening virtual processes, but by evaluating the current hierarchies and structures that trickle down into mass-market AR technologies. The last is through application of current theories of radical enactivism / phenomenology in order to examine the experience of embodiment in AR processes.

Chapter four contains the methodology through which I address my research questions, this is separated into practice, study, data collection, and analysis methods. The first, practice methods, outlines the design principles for my approach to developing rich multisensory AR experiences (MSAR Experiences) through MSAR Instruments and MSAR Environments. These experiences vary in sensory intensity, engagement methods, and content. Snippets describe small-scale clip-like experiences, whereas Scenes describe medium-scale experiences, and Spaces in-turn describe experiences that take up the space of a room and involve an audience rather than a sole user. The second, reviews the study methods these AR experiences pass through in chapters five to seven; ranging from autobiographical design, to individual studies implementing iterative design, to a group audience study in the form of a large-scale AR installation. The last two sections reviews the data collection and analysis methods applied across the three studies.
 
In chapter five, I outline the development and evaluation of the ‘area~’ system. ‘area~’ enables users to record, manipulate, and spatialise virtual audio samples or nodes around their immediate environment. Through a combination of ambisonics audio rendering and hand gesture tracking, this system calls attention to the ability of non-visual AR, here, audio AR, to provide new aesthetic experiences of real and virtual environments. Through an autobiographical design study, this pilot study proposes that rich experience can result from non-visual AR systems. 

Chapter six [study under construction]

Chapter seven [study under construction]

Chapter eight briefly reviews my theoretical framework and research methods before demonstrating the take home findings of my studies in order, and relative to my research questions. It also contextualises findings in the bigger picture of each research field. It details the contributions of the thesis, and offer concrete examples of how knowledge and practice has changed due to these contributions. It then provides future research areas related to the studies, and details any limitations of the methods and studies.
