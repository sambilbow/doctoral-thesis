% ---------------------------------------------
%  _       _
% (_)_ __ | |_ _ __ ___
% | | '_ \| __| '__/ _ \
% | | | | | |_| | | (_) |
% |_|_| |_|\__|_|  \___/
% ---------------------------------------------
%*[ ]   re-draft
% ---------------------------------------------
\chapter{Introduction}
\label{sec: introduction}
\markboth{}{Introduction}
\epigraph{\emph{I dream of instruments obedient to my thought and which with their contribution of a whole new world of unsuspected sounds, will lend themselves to the exigencies of my inner rhythm}}{\citep{varese1966}}
% ---------------------------------------------
\section{Multisensory AR as a Medium for Embodied Musical Performance}\label{sec: introduction-summary}
In the last twenty years, computational technology has become more and more expressive, the systems we engage with becoming increasingly interactive. Due to this, the arts (especially forms of sound-driven digital art) have embraced new technologies in tandem with, and often contributing to their development. This has lead to more human-centred methods of designing tools of digital art creation, moving away from black-boxed (undisclosed to the audience) interactions, towards more performative and suggestive interactions, which often include the audience. Among some of the technologies that have seen nascent in the arts is XR (the X stands for nothing and everything, the R stands for Reality), a group of technologies that promise to radically rethink the way we approach reality. 

The category of XR comprises the technologies of: Virtual Reality (VR) - technology that promises to immerse us in a completely synthetic virtual world; and Augmented Reality (AR) -  technology that promises to immerse us in our own real world through the computational mediation of virtual processes. This leads to the augmentation, diminution, hybridisation, or extension of our own experience of reality, blurring and perhaps dissolving the line between our virtual and real selves. The present thesis makes no distinction between `Mixed Reality' (MR) and Augmented Reality, and adopts AR as an umbrella term for all contemporary (and mostly marketing) uses of the term MR.

AR, as mentioned, has seen sporadic use in the arts in the last twenty years. Its use in other disciplines, such as medical science, manufacturing and repair, annotation and visualisation, and the military \citep{azuma1997}, have all been typically characterised by its use as technology that aids in the \textbf{visual overlay and alignment} of \textbf{virtual graphics} (text, image, animation, video) onto our \textbf{real world environment}. Within these fields, this has a potential myriad important uses, from training junior doctors in dummy procedures by overlaying the steps of a difficult surgery, to allowing three-dimensional networked collaboration across vast distances. Within the arts, its use has also been typically limited to this visual overlay approach. Despite being only a fraction of the possible forms of AR, this view of AR dominates the market of consumer AR devices, from head-mounted, to handheld and projective technologies, the majority deal with overlaying visual information.  By taking a do-it-yourself (DIY) approach to designing and implementing custom digital musical tools through hardware and software, the present practice-based research thesis approaches AR through a multisensory lens to designing rich, expressive experience. The importance of such an approach is in its ability help redefine, and understand more about the technologies with which we are choosing to mediate our daily lives with, and also about how our own perception of digital media is affected through this embodied usage. 



% ---------------------------------------------
\section{Personal Motivations}\label{sec: introduction-motivations}
My own motivations for taking this multisensory, embodied approach to designing for AR is in the potential expressive affordance of the tool in creative rich, sensory artwork, which seeks to teach us more about the way we perceive ourselves, others, and our environment. Coming from a background in music technology, specifically in developing and evaluating digital music tools and instruments, the nature of typical `visual overlay' AR is incompatible with its use as a digital music tool, and as such, it seems not only possible, but reasonable for it to be redefined to encapsulate all of the senses. In wanting to foster embodied experience between audience members (collaborative expression), AR offers a unique way of mediating and co-constructing (with them) the space in which they embody being and knowing. Thus, my motivation is also within the advancement of our understanding this hybrid space, in which unique relationships between virtual and real processes are played out.



% ---------------------------------------------
\section{Key Research Questions}\label{sec: introduction-researchquestions}


AR has seen nascent implementation as a creative medium in the arts, and more specifically music composition and performance. Therefore, the research areas and questions for the present thesis are based in the intersection between the arts, technology, sensory perception:

\RQallsub

% ---------------------------------------------
\section{Aims \& Objectives of the Thesis}\label{sec: introduction-aims}
As well as answering and contributing understanding to the above research questions, this thesis aims to:

\begin{itemize}
    \item Conduct a thorough literature and practice review of the field of AR within the arts
    \item Contribute to the understanding of the `tools of use' that arise from ARs application within the arts
    \item Apply theories of space, materiality, and embodiment to contemporary AR design, use, and evaluation.
    \item Contribute to the practice field of AR arts by designing and evaluating multisensory AR tools for creative and collaborative expression.
    \item Contribute to the understanding of how such tools might affect our perceptions of self, others and our real, virtual, and hybrid environments. 
\end{itemize}



% ---------------------------------------------
\section{Research Methods}\label{sec: introduction-methods}
The research methods used to answer my research questions, and achieve my aims and objectives above involve taking a practice-based approach to designing multisensory digital musical instruments using AR. Involved too, will be qualitatively (coded interview), and quantitatively (survey of experience) evaluating these instruments and tools, and iteratively redesigning them to provide rich and immersive experience.




% ---------------------------------------------
\section{Knowledge and Aesthetic Experience}\label{sec: introduction-aestheticexperience}
What is experience when it comes to art; how does it relate to epistemological practice? To address this question, the present thesis draws from the work of the American pragmatist John Dewey (1859-1952). In his 1934 text, Art as Experience, Dewey argues that the production and consumption of art faces a crisis. After having been historically coupled in an constitutive relationship with the processes of everyday sociocultural life, it is being increasingly separated from these ``conditions of origin and operation in experience" \citep{dewey1934}, in a way that places art upon a remote pedestal. Dewey describes this as a product of the growth of capitalism. On proposing what they term Dewey Aesthetics, Leddy and Puolakka write:
\begin{quote}
    ``Nothing about machine production per se makes worker satisfaction impossible. It is private control of forces of production for private gain that impoverishes our lives. When art is merely the `beauty parlor of civilization,' both art and civilization are insecure. We can only organize the proletariat into the social system via a revolution that affects the imagination and emotions of [hu]man[kind]. Art is not secure until the proletariat are free in their productive activity and until they can enjoy the fruits of their labor. To do this, the material of art should be drawn from all sources, and art should be accessible to all." \citep{leddy2021}
\end{quote}
From this standpoint, art can serve as an emancipatory force for positive social change; but only on the condition that it is first brought back to the ``origin and operation" of everyday experience — through the democratisation of a wider corpus of artistic media, tools, and social contexts in which these are deployed. In the 21st century, we could mistake this for already having happened. The increasing availability of ubiquitous technologies such as the internet, wearables, smartphones, powerful computers and software have shifted artistic production closer to the site of everyday sociocultural life — from the studio to the bedroom. Yet in doing so, has art really been knocked from its pedestal and been re-integrated into, and resituated to arise from everyday life? 

Today, the fabric through which art tends to be disseminated and therefore consumed, social media, arguably operates within this same capitalist framework, the vocabulary having shifted from art in the museum, to content on our feeds. Despite making art more accessible to produce and consume, the capitalist logic of surplus extraction the centre of the algorithms that determine our interaction with social media vies to keeps art separate from everyday life. Central to this is that instead of being centralised in a museum or gallery, it is the concept of the pedestal itself that we should venerate. We must each curate our own museums and galleries, our own feeds.

The unavoidable truth of these platforms however, is that they are not solely focused on fostering individual or even collective curation of democratised art/content. They still employ the capitalist framework at their core. This extractivist profit motive defines the algorithmic fabric of online ``content creation" (production), and resultant social media ``engagement`` (consumption) through mass data harvesting and advertisement selling. The longer users are engaged on a platform (Facebook, Instagram, Twitter, YouTube, TikTok), the more likely they are to generate profits for the platform via advert click/tap-throughs. Our feeds are interspersed with adverts and sponsored posts that have been carefully curated to maximise the potential click/tap-through rate of their victims. Soshana Zuboff defines this as arising from a ``market of human behavioural futures" [p?] in her book Surveillance Capitalism. Through this surveillance, large amounts of behavioural, affective, and personal data (termed behavioural surplus by Zuboff) are `skimmed' off the top of our engagement with these platforms, and sold to agencies that match this personal data with products/content you are most likely to engage with. How could it be argued that that the production and consumption of art / content on such platforms is not governed by ``private gain" at our expense? 

More recently, there are facets of the rise of NFT art projects that epitomise this continued fetishisation and veneration of digital fine art under the guise of ``decentralisation". Unfortunately enough, these projects often fall foul to the exact ``centralisation" they market to oppose, creating small micro-communities of inter-centralised followers that collectively and actively, through social media, ensnare new and uninitiated retail investors via the urgency phenomenon of the ``fear of missing out". The motivation being to increase the value of their own NFTs. In most NFT projects, the only drive is profit motive, the digital art is merely the vehicle through which this investment is (im)materially realised. Despite the technological basis of some of these projects being quite radical in their proposition of democratic ownership and the privacy and security of digital information, the asset class most of them end up facilitating is tied to a pyramid scheme that necessitates the enrichment of the creators and early adopters of that specific NFT collection. How, in any way, could this method of production and consumption be said to be different from Dewey's outlining of the rise of the ``nouveaux riches":
\begin{quote}
    ``The nouveaux riches, who are an important by-product of the capitalist system, have felt especially bound to surround themselves with works of fine art which, being rare, are also costly. Generally speaking, the typical collector is the typical capitalist. For evidence of good standing in the realm of higher culture, he amasses paintings, statuary, and artistic bijoux, as his stocks and bonds certify to his standing in the economic world." \citeyearpar[p. 7]{dewey1934}
\end{quote}
It could be argued therefore, that we have witnessed an increase in the amount of new media formats, but not one that resituates production outside of capitalist logic for the masses; even in so called ``decentralised" projects. When Leddy and Puolakka stress that it is this ``private control of [the] forces of production for private gain" that leads to this disconnection of art from experience, it follows that the creation of art outside of the confines of this control addresses the imperative for the drawing of art ``from all sources" [own emphasis]. From this view, the convergence of digital art to their ``conditions of origin and operation in experience" could be seen as one of the primary foci of more recent movements such as Maker, Hacker, and DIY Labs, as well as the general ethos of participatory design and open-source hardware and software in art, design, and electronic music practice.

But what does it mean to say that art ought to originate and operate ``in" experience, and how might these movements address this? Dewey considers the ``live creature" in response to this question. He views the participant of aesthetic experience as meaningfully inseparable from the environment in which they are embedded: 
\begin{quote}
    ``The senses are the organs through which the live creature participates directly in the on-goings of the world about [them]. In this participation the varied wonder and splendour of this world are made actual for [them] in the qualities [they experience]. This material cannot be opposed to action, for motor apparatus and ``will" itself are the means by which this participation is carried on and directed. It cannot be opposed to ``intellect", for mind is the means by which participation is rendered fruitful through sense; by which meanings and values are extracted, retained, and put to further service in the intercourse of the live creature with [their] surroundings." \citeyearpar[p. 22]{dewey1934}
\end{quote}
From this basis, namely that perception, sensorimotor action, and intellect cannot be meaningfully separated when considering the intercourse of participants with their environment, Dewey highlights the weakness in the traditional dualist theory of mind -  we are embodied and embedded beings. He posits that experience results from the ``interaction of organism and environment",  and that in its fullest, experience can represent a ``transformation of interaction into participation and communication". It would follow, therefore, that in the production of artistic works, consideration of this dynamical relationship that constitutes our subjective experiences is not only valuable, but that artistic works should aim to arise (originate) from common and relatable states of this relationship.  Returning art to this origination and operation ``in" experience through interactive and participatory digital means can therefore be seen as a crucial mechanism for inducing positive societal change through fostering new channels of ``participation and communication". Shifting the production of artistic work away from the exploitative practices of mass manufacturing inherent in consumer technologies, as well as centring embodied participatory design practices, is a common theme among DIY and Makerspace communities.

This is not to in any way demean or reduce the social importance of art that does not take these approaches. It is also especially tricky to prove in any certainty the subjective measures of aesthetic experience that could result in positive societal change. However, I do view the imperative to shift production and consumption of art outside the capitalist architecture of current proprietary software tools and consumer technologies, and bringing the aesthetic closer to every day experience as a fruitful endeavour for the artist. I argue that doing so would emphasise the dynamical relationship participants find themselves in with their and, crucially, others' sociocultural, economic, and ecological contexts.

Where does knowledge lie for us as researchers in the production and consumption of such artistic works? On what basis can we test and evaluate these claims? One view that is popular in the field of music technology is that of considering the nature of the interactive and digital systems that we develop; a consideration of the epistemic tool. 
\begin{quote}
    ``The digital instrument is an artefact primarily based on rational foundations, and, as a tool yielding hermeneutic relations, it is characterised by its origins in a specific culture. This portrayal highlights the strengthened responsibilities on the designers of digital tools, in terms of aesthetics and cultural influence, as they are more symbolic and of compositional pertinence than our physical tools." \citep[p. 335]{magnusson2009a} 
\end{quote}
Developed by Magnusson, this line of thinking proposes that the digital instruments we design and employ as artists have inscribed in their physicality, affordance and sonic output, knowledge of cultural, historical, and designerly significance. In this way, they can be considered complex systems of research importance through the examination of their symbolic design, construction, performance, and appreciation by audiences. Magnusson embeds this proposition within Don Ihde's philosophy of `instrumental realism' and Wittgenstinian philosophy, and draws from an understanding of the extended mind hypothesis. This theory, which will be expanded on in the next section, proposes that 
\begin{quote}
    ``[Under certain conditions], the human organism is linked with an external entity in a two-way interaction, creating a coupled system that can be seen as a cognitive system in its own right." \citep[p. 7]{clark1998}
\end{quote}
Similar to Dewey's conception of how the live creature is inseparable from the environmental conditions that it is embedded in, the extended mind hypothesis and similar theories of embodied cognition prove to be invaluable in the field of tangible user interfaces, and digital music instrument design, due to the way they can explicate as well as draw out the dynamic relationships between artist, collaborator, material, interface, and audience in the deployment of expressive artistic media. In the following sections, I shall dive deeper into these theories of mind (ToM), and what they have to offer a practice that makes use of augmented reality technologies within the field of computational art and musical performance.



% ---------------------------------------------
\section{Outline of Chapters}\label{sec: introduction-outline}
Chapter two starts with an in-depth history of AR as an emerging technology within the field of Human Computer Interaction (HCI), examining the historical barriers to usage such as portability, price that have led towards its popularity in a variety of forms today. Among these forms, the majority deal with simple and descriptive visual additions in the form of text, image, animation or video overlay. I make a case for `multisensory' AR, i.e. the designing of content or experience for multiple senses to counter not only this bias toward visual design, but also as a method of attaining richer and more novel experiences of AR. This case is made by examining current advancements in multisensory HCI applications and enabling technologies. In the second part of the chapter, I posit through a review of contemporary art practice, the usefulness of AR use within the arts in contributing to the designing and evaluation of these tools, enabling new aesthetic and multisensory experiences, collaborative expression, and agency.

In chapter three, I identify three lenses through which to view, not only the potential of modern-day applications of multisensory AR art, but also the potential of the medium itself. The first is through considering the intervention of AR processes in real and virtual spaces, how this effects the construction of these spaces, the implications this in-turn has for the type of art that is created, and the privacy and security of people co-located in these spaces. The second is through a consideration of the materiality and form of AR processes for both designer, user, and audience; not only by focusing on the tangibility and aesthetics of intervening virtual processes, but by evaluating the current hierarchies and structures that trickle down into mass-market AR technologies. The last is through application of current theories of radical enactivism / phenomenology in order to examine the experience of embodiment in AR processes.

Chapter four contains the methodology through which I address my research questions, this is separated into practice, study, data collection, and analysis methods. The first, practice methods, outlines the design principles for my approach to developing rich multisensory AR experiences (MSAR Experiences) through MSAR Instruments and MSAR Environments. These experiences vary in sensory intensity, engagement methods, and content. Snippets describe small-scale clip-like experiences, whereas Scenes describe medium-scale experiences, and Spaces in-turn describe experiences that take up the space of a room and involve an audience rather than a sole user. The second, reviews the study methods these AR experiences pass through in chapters five to seven; ranging from autobiographical design, to individual studies implementing iterative design, to a group audience study in the form of a large-scale AR installation. The last two sections reviews the data collection and analysis methods applied across the three studies.
 
In chapter five, I outline the development and evaluation of the ‘area~’ system. ‘area~’ enables users to record, manipulate, and spatialise virtual audio samples or nodes around their immediate environment. Through a combination of ambisonics audio rendering and hand gesture tracking, this system calls attention to the ability of non-visual AR, here, audio AR, to provide new aesthetic experiences of real and virtual environments. Through an autobiographical design study, this pilot study proposes that rich experience can result from non-visual AR systems. 

Chapter six [study under construction]

Chapter seven [study under construction]

Chapter eight briefly reviews my theoretical framework and research methods before demonstrating the take home findings of my studies in order, and relative to my research questions. It also contextualises findings in the bigger picture of each research field. It details the contributions of the thesis, and offer concrete examples of how knowledge and practice has changed due to these contributions. It then provides future research areas related to the studies, and details any limitations of the methods and studies.
