% ---------------------------------------------
\chapter{A DIY Approach to AR in the Arts}
\label{sec: method}
\markboth{}{A DIY Approach to AR in the Arts}
\epigraph{\emph{This chapter draws some of its content from a published short-paper in the Doctoral Consortium of the Tangible, Embedded, and Embodied Interaction Conference 2021}}{\citep[]{bilbow2021b}}
% ---------------------------------------------
% lr > theory > YOU ARE HERE > studies > discussion > conclusion


%\quickwordcount{content/4_Methodology}{Methodology}
\section{Outline of Methodological Approach}\label{sec: method-outline}
\subsection{Aims \& Objectives}\label{sec: method-outline-aims}
In \autoref{sec: literature}, the historical origins, contextual trends, exceptions, and contemporary forms of AR technologies were outlined. Additionally, examples of AR's use in the arts were provided, along with specific rationales for their use, e.g. sensory engagement, collaborative expression, and activism and agency. In \autoref{sec: theory}{the previous chapter}, I outlined a trio of lenses through which to consider the material, embodied, and spatial aspects of AR. These drew from a 4EC approach to experience, which posits that cognitive processes are embodied, embedded, enacted and extended. In the current chapter I will be outlining my methodological approach to designing, experiencing, and evaluating musical AR experiences, and AR DMIs. A review of the outstanding aims and questions of the thesis will be first outlined, to provide the chapter a clear direction into the ensuing three study chapters.

\begin{enumerate}
    \RQmedium
    \RQexperience
    \RQfuture
\end{enumerate}




%rationale here too > ground in holistic theoretical understanding of ar
%addobjs
%contributions

%vvvv how ure gonna do it
\subsection{Research Methods}\label{sec: method-outline-methods}
The present thesis approaches research by making use of several established research methodologies. Broadly, to address the research questions outlined above, the earliest conception of the methodological approach of this thesis proposed the creation and iterative design of multisensory AR experiences, and a set of reliable design patterns for doing so. These design patterns started out as a spectrum of artistic AR experience size and complexity: Snippets, Scenes, and Spaces, and later expanded to include approaches to artistic AR workflow, a categorisation of AR sensory displays, and a collection of usable software environments. This is detailed in \autoref{sec: method-patterns}. 

The methodology present in this thesis has stayed consistent with its original proposal, with the exception of specific methods of data collection and analysis being disrupted by the COVID-19 pandemic, which commenced three months after the start of research. started as creating smaller (in complexity and in space) experiences, that were simple in their sensory modulation and instrumental reactivity, and later incorporate them in larger scale interactive AR experiences (Spaces) once sufficiently tested. However, due to COVID-19 the prospect of user-testing and iterating on the design of multiple experiences, especially large scale ones, through participant feedback became increasingly difficult to facilitate and plan for. Thus, I opted instead to focus on developing working musical AR prototypes that I myself would play regularly, tweak, and iterate upon. Then, and there, an AR research and compositional practice began.

\subsection{Studies}
\subsubsection{\autoref{sec: area} -  The area\textasciitilde{} System}
The starting point for this was the creation of area\textasciitilde{}, a non-visual audio AR experience centred around the self-composition of a hybrid listening environment using an ambisonic microphone, head-, and hand-tracking. As will be expanded upon in \autoref{sec: area}, due to the circumstances of the COVID-19 lockdown, I opted for an autobiographical design method for the iteration and evaluation of the system. Out of the evaluation of this project, namely the endeavour to expand into audiovisual AR experiences, I discovered a suitable platform the develop the rest of doctoral research on: the open-source Project North Star AR headset.

\subsubsection{\autoref{sec: polaris} - Evaluating polaris\textasciitilde{}}


\subsubsection{\autoref{sec: performance} - Experimental A/V AR Performance @ The Rosehill}


\subsection{Documentation of Practical Work}\label{sec: method-outline-documentation}
Practical work, where carried out, has always been documented for analysis, open-research, and archival purposes. I found the combination of my personal website as well as GitHub a suitable solution for this.



\section{Practice-based Resistance}\label{sec: method-resistance}
Formative in the development of the methodological approach of this thesis has been the concept of `resistance'. In short, this could be characterised as a motivation for changing or acting against the state of the art, practice, technology, or philosophy, born out of a discontent for its status quo. This isn't to be confused by resistance in the sense of experiencing resistance to do something, although as might be expected, that has reared its head a few times over the last three years too. Its difficult to place resistance into the linear narrative of this thesis, some resistance formed my initial motivations for the carrying out of this research, and some was garnered along the way, especially around the time of the uptake in `Metaverse' as a term to describe the sum total of all XR development. Either way, resistance has made up a large portion of my perspective on AR technologies, its role within digital music and humanities more broadly, and my rationale for pursuing the specific research methods found in the thesis.

\subsection{Maker, DIY, and Hardware Hacking Subcultures}\label{sec: method-resistance-maker}
As referred to in \autoref{sec: theory-aestheticexperience}, contemporary design research in the fields of computational art and music have in recent years rallied behind the broader ``free/libre and open-source software" and ``open-source hardware" ethea (abbreviated to FLOSS and OSH respectively). This approach, often hand-in-hand with the Maker, DIY, and Hardware Hacking subcultures, stands in stark contrast to the goings-on of mega-corporations board rooms, with their design `sprints', `agile' workflows, and `human-centred' design. While commentary on the considerable role of un(der)funded volunteer labour involved in OS tools and projects is outside of the scope of this thesis, it is worth mentioning. The approaches taken by these subcultures towards embodied design and composition workflows however, has been highly beneficial. In the case of digital music making, software tools and coding languages like PureData, Maximilian, Supercollider, ixilang, Sonic Pi, TidalCycles, and Bela have granted students, researchers, and composers across disciplines, free and beginner-friendly access to a wealth of tutorials, examples, and tools for advanced real-time sampling and synthesis techniques.

It could be proposed, that this approach is motivated in part by a resistance against many of the `features' typically found in commercial ``closed-source" software and hardware. Providing significant barriers to accessible learning, creativity, and artistic authenticity, these include lack of interoperability with other software, use of proprietary file formats, high initial or recurring subscription costs, and restricted access to lower-level parameters, functions, and settings. It is within this resistance against the closed-source and / or commercial that the practical work of this thesis attempts to situate itself, drawing from its implicit designerly knowledge and aesthetics. As a result, the path taken to address the key areas of the thesis' scope involve the drawing out of knowledge from the creative practice of hacking and making, designing and iterating on experimental and DIY hardware and software.

\subsection{AR as a visual experience}\label{sec: method-resistance-visual}
A second form of resistance that the methodological approach of this thesis draws on is the ocularcentrism found in AR technologies mentioned previously in \autoref{sec: literature-interface-sensory-visual}. As a practitioner of music, and audio related technologies, this has provided a significant portion of the motivation for carrying out the present research, as well as a significant portion of the challenges faced when attempting to develop AR as a medium for sound-based interactive experiences. %% wanting to make room for later ms endeavours, time/expertise/cost/access barrier for the moment

\subsection{AR as an additive process}\label{sec: method-resistance-additive}
%% art is inherently hybrid, the space counts etc

\subsection{Consumption-driven experiences}\label{sec: method-resistance-consumer}
%% art as a commentary on consumption/stop time?


\section{Design Patterns for AR Art and Music} \label{sec: method-patterns}
% The XRt Space
% guidelines for accessibility etc
\subsection{Interaction}
\subsubsection{Snippets}
\subsubsection{Scenes}
\subsubsection{Spaces}

\subsection{The Experience}
\subsubsection{The AR Subform}
\subsubsection{The Real-Virtual Dynamic}
\subsubsection{Sensory Engagement}

\subsection{The Instrument}
%display
\subsubsection{Wearable}
\subsubsection{Tangible}
\subsubsection{Situated}

\subsection{The Environment}
\subsection{Balance between R/V}
\subsection{Allowance for the Real}

