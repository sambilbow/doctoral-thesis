% Set page style to empty
\newpage
\phantomsection
\addcontentsline{toc}{section}{Foreword}
\begin{flushleft}
	\Huge \textsc{\textbf{Foreword}}
	
\end{flushleft}

\noindent Since beginning my academic journey eight years ago, the world -- both its real and virtual counterparts -- seems to have made tectonic shifts. There have been five conservative prime ministers here in the UK, and any reasonable discussion or policy around the most pressing matters facing the lives of people around the world -- especially those who are most effected -- has been all but absent. The COVID-19 pandemic, which began three months into this research, has only made it more painfully obvious that business-as-usual is not an option if we want a brighter and fairer future. Global supply-chains are hegemonic and fragile \citep{gomez2020}, Western governments' health policy is naïve \citep{navarro2021}, their climate policies purposefully inactive \citep{slawinski2017}, leading to evermore demonstratively short-sighted mistreatment of nature and non-human animals  \citep{monbiot2022}. Only last week, the conservative government green-lit a \pounds165 million deep coal mine in Cumbria, whose coal output could release 220 million tonnes of CO$_2$e downstream, over the course of the mine's 25 year lifetime \citep[Grubb and Barrett in][p. 252]{grubb2022}.

In the digital world -- if it can at all be meaningfully separated from its physical counterpart any more -- the last eight years have been equally tumultuous. From the Facebook / Cambridge Analytica scandal \citep{isaak2018}, to the rise of troll-farms threatening online and offline misinformation \citep{badawy2018}; more and more, our understanding about the world and the people around us seems to be mediated by technology delivered by complicit or at the very least, apathetic mega-corporations. In spite of the convenience, interconnectedness and numerous other positives social networks have fostered, it has meant that in turn, the keys to our own behaviour are within reach of those with the capital to access and purchase it \citep{zuboff2019}. This is compounded by the fact that centralised social networks also tend thrive on \citep{thorleifsson2022}, amplify \citep{mathew2019}, and promote \citep{ccdh2022,adl2022}  -- misogynistic, white-nationalist, antisemitic, homophobic, transphobic, and ableist hate speech, that has profound micro (individual), meso (group) and macro (societal) implications \citep{alkiviadou2019}. 

It was in front of this grey backdrop that I began the journey of a PhD four years ago. Despite (or due to) growing up with and around social media since the age of 11 -- and economic, social, and cultural privilege notwithstanding -- this backdrop has had a profoundly negative impact on my mental health since the COVID-19 started three months into this journey. More recently this effect was compounded by the threat of global nuclear war.
\clearpage

\phantomsection
\addcontentsline{toc}{section}{Acknowledgements}
\begin{flushleft}
	\Huge \textsc{\textbf{Acknowledgements}}
	
\end{flushleft}

\noindent I therefore cannot express enough how much gratitude and appreciation I feel for my partner and fiancée Hanna for the amount of support she has so lovingly provided me over the last few years. She has made the tough days bearable, and the fun days memorable, I simply couldn't have done it without her. The past three months would not have been possible without the help of my therapist too; thank you Sarah. I would also like to thank my friends for encouraging me to take on this journey and providing me with more support than I think they know, especially Dan, Glory, Lauren, Joe, Dash, Chan, Nik, Harry, and the BeFries Team; and also those I've met along the way who have provided helpful ears, Fiona, Yaz, Colin, Katherine and Chris. Special thanks too to my family, close and extended who have provided me with motivation in times when I lacked the ability to conjure up my own.

The support from my main supervisors, Chris and C\'ecile has been invaluable, and I am indebted to their guidance over the past eight and three years respectively. Thanks also to Marianna, who provided me with insight on the role of multisensory technology at the start of the project, and Jamie, who took up the mantle in 2020, and has since offered insight into the cognitive neuroscience of perception.

I have also had the honour of sharing time and creative work with a number of Sussex faculty over the past eight years who are also very much deserving of thanks for their inspiring feedback on my work, specifically Thor Magnusson, Alice Eldridge, Dylan Beattie, Danny Bright, Alex Peverett, Joe Watson, Evelyn Ficarra, and Anil Seth. In addition, I have had the pleasure of meeting, making friends, and collaborating with, artists and researchers along the way. Thank you to Jon, Sissel, Steve, Dimitris, Max, Halld\'or and the Leverhulme DSP second cohort.

This research has made use of numerous free / libre open source software and hardware projects and communities, and wouldn't have been possible without the hard work and dedication of those employed at Leap Motion and Ultraleap: David Holz, Florian Maurer, Johnathon Selstad, Adam Hardwood, Pip Turner, and Alex Colgan; and Project North Star community maintainers: Noah Zerkin, \textasciitilde{}j0ule, Damien Rompapas, Bryan Chris Brown, Juraj Vincur, Nova, and Moses to name but a few. Additionally, I have been made to feel welcome in several academic communities, including AudioMostly, TEI, and NIME, so thank you to the countless community organizers, and peer-reviewers that ensure that research in the sonic arts is a fun and welcoming endeavour. Thank you, lastly, to all of the pilot and study participants.

This doctoral research was funded by the Leverhulme Trust Doctoral Scholarship Programme: "Sensation and Perception to Awareness". Directed by Prof. Anil Seth, and Prof. Jamie Ward.
