% % Abstract

\thispagestyle{empty}
\pdfbookmark[0]{Abstract}{Abstract} % Bookmark name visible in a PDF viewer

\begin{center}
%	\bigskip

    {\normalsize \href{http://www.sussex.ac.uk/}{\myUni} \\} % University name in capitals
    {\normalsize \myFaculty \\} % Faculty name
    {\normalsize \myDepartment \\} % Department name
    \bigskip\vspace*{.02\textheight}
    {\Large \textsc{Doctoral Thesis}}\par
    \bigskip
    
    {\rule{\linewidth}{1pt}\\%[0.4cm]
    \Large \myTitle \par} % Thesis title
    \rule{\linewidth}{1pt}\\[0.4cm]
    
    \bigskip
	{\normalsize by \myName \par} % Author name
    \bigskip\vspace*{.06\textheight}
\end{center}

    {\centering\Huge\textsc{\textbf{Abstract}} \par}
    \bigskip



    \noindent This thesis resituates augmented reality (AR) as an artistic medium for the creation of interactive and expressive works by computational artists and musicians. If an AR system can be thought of as one that combines real and virtual processes, is interactive in real-time, and is registered in three dimensions; why do we witness the majority of AR applications utilising primarily visual displays of layered information? 
    
    In this practice-based research, I propose a compositional approach to developing sound art using AR (sound ARt), arguing that, as an medium that combines real and virtual multisensory processes, it must explored with a sensory-process agnostic approach. After
    
    Three sound ARt experiences are outlined, and embody the majoirty of the practical contributions of this thesis: \textit{\hyperref[sec: area]{area\textasciitilde{}}}, \textit{\hyperref[sec: polaris]{polaris\textasciitilde{}}}, and \textit{\hyperref[sec: polygons]{polygons\textasciitilde{}}}. 
    
    In discussing the results of the three study chapters, three theoretical propositions, termed: \hyperref[sec: discussion-medium-material]{augmented materiality}, \hyperref[sec: discussion-medium-embodiment]{augmented embodiment}, and \hyperref[sec: discussion-medium-space]{augmented space} are outlined.
    
    Out of the iterated design, studies, evaluation, and discussion, design patterns for those in the field interested in reproducing or developing similar sound ARt have been developed: \textit{\nameref{sec: discussion-patterns-experience}}, \textit{\nameref{sec: discussion-patterns-instrument}}, and \textit{\nameref{sec: discussion-patterns-environment}}. 
    
    Further outlines the study methods that I will use to evaluate the developed experiences. The contributions of this project are twofold, the development of a set of design patterns for creating sound ARt, as well as AR hardware, software and three sound ARt experiences that have been developed and evaluated.