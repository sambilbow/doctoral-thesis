\phantomsection
\addcontentsline{toc}{chapter}{Thesis Summary}
\begin{flushleft}
	\Huge \textsc{\textbf{Thesis Summary}}
\end{flushleft}
\begin{flushright}
    {\normalsize \textsc{A Doctoral Thesis in Music Technologies by \myName}\\}
    {\normalsize \textsc{\myUni, Falmer, East Sussex, BN1 9RG} \\} % University name in capitals
    {\normalsize \textsc{\myFaculty} \\} % Faculty name
    {\normalsize \textsc{\myDepartment}} % Department name
\end{flushright}
\begin{flushleft}
    \Large \myTitle \\% Thesis title
    \normalsize \mySubtitle \\
\end{flushleft}
\begin{SingleSpace}
    It has been thirty years since the original definition of \glshyperlink[augmented reality (AR)]{ar} as a technology used to `augment the visual field of a user with information necessary in the performance of tasks'. In this first instance, it was developed with the purpose to `improve the efficiency and quality of human workers in their performance of manufacturing activities' \citep{caudell1992}. Alongside subsequent decades of funding from the U.S. \glshyperlink[military-industrial complex (MIC)]{mic}, we have also seen the uptake and reappropriation of \glshyperlink{ar} in creative fields, such as computational art, performance, design, and entertainment - these works often proposing \glshyperlink[do-it-yourself (DIY)]{diy} and open-source approaches to their design. 

    Despite these developments, \glshyperlink{ar} within sound-driven forms of art have been relatively under-explored. If an \glshyperlink{ar} system can be thought of as one that can combine real and virtual multisensory processes, is interactive in real-time, and is registered in three dimensions \citep{azuma1997}; why do we, thirty years on, witness the paradigmatic form of \glshyperlink{ar} still being heavily biased \citep{billinghurst2015} towards it being a method of visual information overlay?
    
    Standing in stark contrast to the currently unfolding and hyper-commercialised view of \glshyperlink{ar} -- as defined by the corporate `Metaverse' -- this thesis resituates \glshyperlink{ar} as an artistic medium for the creation of interactive and expressive works by musicians and sound artists. It is guided primarily by the questions: \textit{`What are \glshyperlink{ar}'s affordances as an artistic medium, the resultant experience for participants and audiences (or `immersants') in these experiences, and what might a future corpus of \glshyperlink{ar} digital music instruments look, sound, and feel like?'}
    
    To address these questions, this practice-based of research takes a \textit{DIY Approach to Sound ARt}, arguing that, as an medium that combines real and virtual multisensory processes, it must explored with a sensory-process agnostic approach -- that is, to approach \glshyperlink{ar} as more than mere visual information overlay -- instead as `real-time computationally mediated perception' \citep{chevalier2020}. This has involved making and hacking technology as an necessary aesthetic and political stance against commercial \glshyperlink{ar} technologies in their typical form.
    
    Three sound \glshyperlink[augmented reality art (ARt)]{art} experiences are outlined, and embody the majority of the practical contribution of this thesis: \textit{\hyperref[sec: area]{area\textasciitilde{}}}, \textit{\hyperref[sec: polaris]{polaris\textasciitilde{}}}, and \textit{\hyperref[sec: polygons]{polygons\textasciitilde{}}}. In discussing the results of these three study chapters, theoretical propositions are made: \textit{\hyperref[sec: discussion-medium-material]{`augmented materiality'}}, \textit{\hyperref[sec: discussion-medium-embodiment]{`augmented embodiment'}}, and \textit{\hyperref[sec: discussion-medium-space]{`augmented space'}}, that have implications for the use of \glshyperlink{ar} as a sonic medium. Moreover, out of the iterated design of the \glshyperlink{ar} experiences, their study, evaluation, and discussion, three `design guidelines' for those in the field interested in reproducing or developing similar sound \glshyperlink{art} have been developed: \textit{\hyperref[sec: discussion-guidelines-experience]{\nameref{sec: discussion-guidelines-experience}}}, \textit{\hyperref[sec: discussion-guidelines-instrument]{\nameref{sec: discussion-guidelines-instrument}}}, and \textit{\hyperref[sec: discussion-guidelines-environment]{\nameref{sec: discussion-guidelines-environment}}}.   
\end{SingleSpace}