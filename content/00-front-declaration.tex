\phantomsection
\addcontentsline{toc}{chapter}{Preface}

\phantomsection
\addcontentsline{toc}{section}{Colophon}\label{sec: reading}
 \begin{flushleft}
	\Huge \textsc{\textbf{Colophon}}
	
\end{flushleft}
\begin{SingleSpace}
\noindent This thesis was typeset by \hologo{LaTeXe} to conform with University of Sussex \href{https://www.sussex.ac.uk/rsao/examination}{guidelines}, using a \href{https://github.com/sambilbow/Sussex_PhDThesis}{modified version} of \href{https://github.com/Martin-Jung/Sussex_PhDThesis}{Martin Jung's} and \href{https://github.com/fabriziomiano/phd_thesis}{Fabrizio Milano's} templates. It was compiled with \hologo{pdfLaTeX}, references exported from \verb|Zotero| and typeset with \hologo{BibTeX}, and the terms list is implemented using \verb|bib2gls| and \verb|glossaries-extra|. It was written and edited in \verb|Scrivener|, \verb|Overleaf|, and eventually \verb|Visual Studio Code| with the \hologo{LaTeX}-\scriptsize{\textsc{Workshop}} \normalsize plugin.

\noindent This version of the thesis is best read and / or annotated using a PDF viewer (maybe one day in AR!). It makes heavy use of hyperlinks, hence the reader may benefit from using a viewer that has navigation history, i.e., backwards \faArrowCircleLeft\space, and forwards \faArrowCircleRight\space buttons. This will help with getting back to the text after following a citation or term definition link. Free / libre open-source PDF viewers with this functionality include \href{https://sourceforge.net/projects/skim-app/}{Skim} \faApple\space \footnote{View $\blacktriangleright$ Customize Toolbar $\blacktriangleright$ add `Back/Forward'}, \href{https://okular.kde.org/en-gb/}{Okular} \faLinux \space \faWindows \space \footnote{Settings $\blacktriangleright$ Configure Toolbars  $\blacktriangleright$ Main Toolbar $\blacktriangleright$ Available Actions $\blacktriangleright$ Back in the Document}, and \href{https://sioyek.info/}{Sioyek} \faApple \space \faLinux \space \faWindows \space \footnote{Keyboard $\blacktriangleright$ Backspace / Shift-Backspace}. These three PDF viewers also allow previewing of links within the thesis, meaning you can glance at references and term definitions without leaving the page.

\noindent Preview \faApple\space \footnote{View $\blacktriangleright$ Customize Toolbar $\blacktriangleright$ add `Page History'} and Adobe Acrobat \faApple\space \faLinux \space \faWindows \space \footnote{View $\blacktriangleright$  Show/Hide $\blacktriangleright$ Toolbar Items $\blacktriangleright$ Show Page Navigation Tools $\blacktriangleright$ Show All Page Navigation Tools} also feature navigation history, but unfortunately not link previews. All PDF viewers mentioned feature sidebar table of contents, which this thesis uses comprehensively for ease of reader navigation.

\noindent The hyperlinks in this thesis have been colour-coded to ensure that the reader knows where they're being taken to. Colours have been chosen for their legibility in both light and dark modes, taking into consideration readers who may have deuteranopia (red-green) colour vision deficiency; prioritising the \textcolor{Hyurlcolor}{website}, \textcolor{Hylinkcolor}{page}, and \textcolor{Hycitecolor}{citation} links' colour-contrast difference \footnote{Using advice from \href{https://colorbrewer2.org/\#type=qualitative\&scheme=Dark2}{ColorBrewer 2.0} \href{https://archive.today/oRW5e}{(Archived \faArchive)}}.

\vspace*{0.25cm}
\noindent\textcolor{Hyurlcolor}{Purple} links direct the reader to an external website via the default browser

\noindent\textcolor{Hylinkcolor}{Orange} links direct the reader to another page in the thesis

\noindent\textcolor{Hycitecolor}{Green} links direct the reader to a reference in the bibliography

\vspace*{0.25cm}
\noindent Additionally, terms that occur frequently within the thesis are defined in the \hyperref[main]{Working Definitions} section. Specifically abbreviated terms are, on first use, prefixed by their `long' name, and marked with a \textcolor{Hylinkcolor}{$\circlearrowleft$} symbol. From then on, the reader may click on the abbreviated term as it comes up throughout the thesis to navigate to its definition entry. Starting from \autoref{sec: review}, subsequent uses of common terms won't be coloured or decorated so as not to distract the reader. However, like \textcolor{Hyurlcolor}{website}, \textcolor{Hylinkcolor}{page}, and \textcolor{Hycitecolor}{citation} links, they are navigable via click or tap, previewable via hover in Skim and Okular, or right-click in Sioyek.

It is pertinent to mention that practical work, where carried out, has always been documented for analysis, open-research, and archival purposes. I found the combination of \href{https://sambilbow.github.io}{my personal website} as well as \href{https://github.com/sambilbow}{GitHub} and \href{https://youtube.com/@sambilbow}{YouTube} a suitable and cost-effective solution for this. Links to project files will be included in the study chapters, and cited websites have been archived using \rurl{archive.today} \footnote{\href{https://en.wikipedia.org/wiki/Archive.today}{Wikipedia - Archive.today} \href{https://archive.today/dWUvw}{(Archived \faArchive)}} where possible, with a separate link (denoted by ``(Archived \faArchive)'') included for use if the page has been removed from the internet. 

\newpage
\phantomsection
\addcontentsline{toc}{section}{Timeline}
\begin{flushleft}
	\Huge \textsc{\textbf{Timeline}}
\end{flushleft}

\noindent\begin{tabular*}{\textwidth}{@{}ll@{\extracolsep{\fill}}r@{}}
	Final Submission & \textit{\hspace{1cm}\myLocation} & \today\\
	Defence of Thesis & \textit{\hspace{1cm}Falmer, UK} & 16th November, 2023\\
	Initial Submission & \textit{\hspace{1cm}P\'erigueux, France} & 22nd December 2022\\
	Draft Submission & \textit{\hspace{1cm}Hove, UK} & 9th November 2022\\
	polygons\textasciitilde{} & \textit{\hspace{1cm}Brighton, UK} & 19th February 2022\\
	polaris\textasciitilde{} & \textit{\hspace{1cm}Brighton, UK} & 20th October 2021\\
	area\textasciitilde{} & \textit{\hspace{1cm}Brighton, UK} & 3rd February 2020\\
	Research Begins & \textit{\hspace{1cm}Brighton, UK} & 19th September 2019
\end{tabular*}


\vfill

\begin{flushleft}
	\phantomsection
	\addcontentsline{toc}{section}{Declaration}
	 \begin{flushleft}
		\Huge \textsc{\textbf{Declaration}}
	\end{flushleft}
	
	\begin{flushleft}
		\noindent I, \myName, hereby declare that this thesis has not been and will not be, submitted in whole or in part to another university for the award of any other degree.
	\end{flushleft}
	
	\begin{minipage}{.45\linewidth}
		\begin{flushleft} %\large
			\textit{\myLocation,}\\
			\textit{\today}
		\end{flushleft}
	\end{minipage}
	\hfill
	\begin{minipage}{.45\linewidth}
		\begin{flushright} %\large
			\makebox[2.5in]{\hrulefill} \\
			\myName 
		\end{flushright}
	\end{minipage}\\
	\end{flushleft}
\vspace{1.5cm}
\end{SingleSpace}%