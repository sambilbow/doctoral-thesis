% ---------------------------------------------
\chapter{Performing polygons\textasciitilde{}}
\label{sec: polygons}
\markboth{}{Performing polygons\textasciitilde{}}
\epigraph{\emph{quote}}{\citep[]{bilbow2022}}
% ---------------------------------------------
%              _                            /\/|
%  _ __   ___ | |_   _  __ _  ___  _ __  __|/\/
% | '_ \ / _ \| | | | |/ _` |/ _ \| '_ \/ __|
% | |_) | (_) | | |_| | (_| | (_) | | | \__ \
% | .__/ \___/|_|\__, |\__, |\___/|_| |_|___/
% |_|            |___/ |___/
% ---------------------------------------------
\section{Developing an AR Performance Practice} \label{sec: polygons-developing}
As a direct consequence of witnessing participants enjoyment and play with polaris\textasciitilde{}, I recognised for the first time looking from the outside in, that the system had a larger propensity for fostering learning, virtuosity, and depth of expression than I had originally thought; if only the experience was slightly more complex, the interactions more \textit{nuanced}. The gestures, play, and expression by participants led to a kind of quasi-transhumanist dance, one that struck me as a technologically mediated dialogue between hybrid self and hybrid environment, a delicate balance of agency.

Until this point, I had been primarily focused on the medium as used for the composition of musical pieces e.g. \hyperref[sec: area]{area\textasciitilde{}}, or of installation-like experiences \hyperref[sec: polaris]{polaris\textasciitilde{}}. Performance with AR had remained abstract, especially as, formally, its not my preferred mode of musical expression. However, the balance of agency between participant and system, between self and environment, presented itself during the evaluation of polaris\textasciitilde{} to be an area ready for exploration through performance. 

Most of the software and hardware I had used in the development of polaris\textasciitilde{} was readily transferable to the context of performance. There was no need to change from the combination of PureData, LibPdIntegration, and Unity. However, performance did confront me with new considerations: 
\begin{itemize}
    \item \textit{What elements of my experience would need to be shared with an audience?} 
    \item \textit{What role would my gestures have in describing the experience that I was by now intimately aware of, but would be strange for an audience?} 
\end{itemize}
The following chapter outlines my exploration of a a fifteen-minute experimental audiovisual AR improvisation using an AR performance ecosystem called \textit{Weird Polygons and Hand Noises}, or \textit{polygons\textasciitilde{}} for short.



% ---------------------------------------------
\section{The Composition of Weird Polygons and Hand Noises} \label{sec: polygons-composition}
\subsection{Augmented Material} \label{sec: polygons-composition-material}
%*! Learning => Improvisation

\subsection{Augmented Embodiment} \label{sec: polygons-composition-embodiment}
%*! Exploration => Movement

\subsection{Augmented Space} \label{sec: polygons-composition-space}
%*! Invitation of the audience into a hybrid shared space



% ---------------------------------------------
\section{Performances} \label{sec: polygons-performances}
\subsection{Technical Setup} \label{sec: polygons-performances-setup}
\subsection{The Rosehill} \label{sec: polygons-performances-rosehill}
\subsection{The Attenborough Centre for Creative Arts} \label{sec: polygons-performances-acca}



% ---------------------------------------------
\section{Demonstrations} \label{sec: polygons-demonstrations}
\subsection{MAH Doctoral Conference} \label{sec: polygons-demonstrations-mah}
\subsection{SHL Make \& Create} \label{sec: polygons-demonstrations-shl}



% ---------------------------------------------
\section{The Future of AR Musical Performance}