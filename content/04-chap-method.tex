% --------------------------------------------------------------------------- %
%      _ _                                                _       _           %
%   __| (_)_   _    __ _ _ __  _ __  _ __ ___   __ _  ___| |__   | |_ ___     %
%  / _` | | | | |  / _` | '_ \| '_ \| '__/ _ \ / _` |/ __| '_ \  | __/ _ \    %
% | (_| | | |_| | | (_| | |_) | |_) | | | (_) | (_| | (__| | | | | || (_) |   %
%  \__,_|_|\__, |  \__,_| .__/| .__/|_|  \___/ \__,_|\___|_| |_|  \__\___/    %
%          |___/        |_|   |_|                                             %
%                                                                             %
%                                                    _                        %
%                    __ _ _ __   _ __ ___  _   _ ___(_) ___                   %
%                   / _` | '__| | '_ ` _ \| | | / __| |/ __|                  %
%                  | (_| | |    | | | | | | |_| \__ \ | (__                   %
%                   \__,_|_|    |_| |_| |_|\__,_|___/_|\___|                  %
% --------------------------------------------------------------------------- %
\chapter{A DIY Approach to AR in the Sonic Arts}
\label{sec: method}
%\markboth{}{A DIY Approach to AR in the Sonic Arts}
\epigraph{\emph{}}{}
% --------------------------------------------------------------------------- %
\section{Methodological Approach}\label{sec: method-methods}
In \autoref{sec: review}, the historical origins, contextual trends, exceptions, and contemporary forms of AR technologies were outlined. Additionally, examples of AR's use in the arts were provided, along with specific rationales for their use, e.g. sensory engagement, collaborative expression, and activism and agency. A positionality was established that demonstrates that AR's use in the arts is at odds with its origination in military and industrial applications. In the \autoref{sec: theory}, I outlined a trio of lenses through which we may begin to consider the material, embodied, and spatial aspects of sound art using AR. In examining complex musical material, it was identified that digital musical systems or ``performance ecosystems" offer us fertile conceptual ground for description of material, in the creation and evaluation of rich experiences. The work of researchers approaching experience from an enactivist perspective when working with XR technologies was then outlined. Finally, an examination of the current state of the `Metaverse' as a space to describe XR experience in general was critiqued. These three lenses all drew from an understanding of art \textit{as} experience, and the disconnection between art and everyday life, an alienation that comes as no surprise given its commodity fetishisation by free-market capitalism. Moreover, the lenses were contextualised within a 4EC approach to experience, which posits that cognitive processes are embodied, embedded, enacted and extended; inseparable from the brain-body-environment that they constitute and are situated. In the current chapter I will be outlining my methodological approach that has informed both the theoretical, and practical sides of the research: designing, performing, and evaluating sound art AR experiences (\textit{sound ARt} to borrow from Rhodes \citeyearpar{rhodes2018}). A review of the outstanding research questions will be outlined, to provide the chapter a clear direction into the ensuing three study chapters.

\begin{enumerate}
    \RQmedium
    \RQexperience
\end{enumerate}

This thesis approaches the above questions by making use of several established practice research methods. Broadly, these methods have led to practical contributions that will be explored in the following three study chapters: 
\begin{itemize}
    \item the creation and iterative design of multisensory sound ARt
    \item a set of open design patterns for creating similar sound ARt compositions and performances
\end{itemize}
These practical contributions have functioned both as ongoing research probes, and as platforms for open and reproducible artistic work for the wider experimental music research community. The methods present in this thesis have stayed consistent with their original proposal, with the exception of specific methods of data collection and analysis being adapted to suit the restrictions of the COVID-19 pandemic, which commenced three months after the start of research. 

As will be developed on in \autoref{sec: discussion-patterns}, it is within this context of creating, developing, implementing and iterating on a set of open design patterns for AR art and music that the research methods employed in the following chapters are carried out. This started with a motivation for creating smaller experiences, that were simple in their sensory modulation and instrumental reactivity, with the view to later incorporate them into larger scale interactive AR experiences once sufficiently tested. However, due to COVID-19, the prospect of user-testing and iterating on the design of multiple experiences, especially large scale ones, through participant feedback became increasingly difficult to facilitate and plan for. Thus, I opted to focus on developing working sound art AR prototypes that I myself would compose with regularly, tweak, and iterate upon. Then, and there, an AR research and compositional practice began.

It is pertinent to mention that practical work, where carried out, has always been documented for analysis, open-research, and archival purposes. I found the combination of \href{https://sambilbow.github.io}{my personal website} as well as \href{https://github.com/sambilbow}{GitHub} and \href{https://youtube.com/@sambilbow}{YouTube} a suitable and cost-effective solution for this. Links to project files will be included in the study chapters. Before outlining the specific methods touched on in later chapters, I would first like to touch upon the concept of resistance, and its role in motivating the methodological approach taken in the thesis.



% --------------------------------------------------------------------------- %
\section{Resistance in Practice} \label{sec: method-resistance}
Formative in the development of the methodological approach of this thesis has been the concept of `resistance'. In short, this could be characterised as a motivation for changing or acting against the state of the art, practice, technology, or philosophy, out of a discontent for its status quo. This isn't to be confused by resistance in the sense of experiencing resistance to \textit{do} something, although as might be expected, that has reared its head a few times over the last three years too. It is difficult to place resistance into the linear narrative of this text; some resistance formed my initial motivations for the carrying out of this research, and some was garnered along the way, especially around the time of the unfortunate uptake in `Metaverse' as a term to describe the sum total of all creative XR development and labour. Either way, resistance has made up a large portion of my approach to AR technologies, its role within digital music and humanities more broadly. These forms of resistance can be separated into three main categories.

\subsection{AR as Consumer Technology} \label{sec: method-resistance-maker}
As referred to in \autoref{sec: theory-experience}, contemporary design research in the fields of computational art and music have in recent years rallied behind the broader ``free/libre and open-source software" and ``open-source hardware" ethoses (abbreviated to FLOSS and OSH respectively). This approach, often hand-in-hand with the Maker, DIY, and Hardware Hacking subcultures, stands in stark contrast to the goings-on of mega-corporations board rooms, with their design `sprints', `agile' workflows, and `human (customer)-centred' design. The approaches taken by these subcultures towards embodied design and composition workflows however, has been highly beneficial for the field. In the case of digital music making, software tools and coding languages like PureData, Maximilian, Supercollider, ixilang, Sonic Pi, TidalCycles, and Bela have granted students, researchers, and composers across disciplines, free and beginner-friendly access to a wealth of software tutorials, examples, and tools for advanced real-time sampling and synthesis techniques.

It could be proposed, that this approach is motivated in part by a resistance against many of the `features' typically found in commercial ``closed-source" software and hardware. Providing significant barriers to accessible learning, creativity, and artistic authenticity, these include lack of interoperability with other software, use of proprietary file formats, high initial or recurring subscription costs, and restricted access to lower-level parameters, functions, and settings. It is within this resistance against the closed-source and / or commercial that the practical work of this thesis attempts to situate itself, drawing on its implicit designerly practices and aesthetics. As a result, the path taken to address the key areas of the thesis has involved the bringing into the fore, knowledge from the creative practice of hacking, making, designing, and iterating with experimental and DIY open-source hardware and software.

While commentary on the considerable role of un(der)funded volunteer labour involved in open-source tools and projects is outside of the scope of this thesis, it is worth mentioning. In this way, `open-source' is never taken to describe the best way of carrying out a project, rather I hope to convey that the ideals and features of an open-source ethic are beneficial to the aims of this thesis, vis-à-vis developing the understanding of new technologies within the NIME field, and attempting to manoeuvre what is essentially a minefield of technologies that are only available today because of U.S. defence funding.

\subsection{AR as Visual} \label{sec: method-resistance-ocularcentrism}
A second form of resistance that the methodological approach of this thesis explored is the ocularcentrism found in AR technologies mentioned previously in \autoref{sec: ar-sensory-visual}. As a practitioner of music, and audio related technologies, this has provided a significant portion of the motivation for carrying out the present research, as well as a significant portion of the challenges faced when attempting to develop AR as a medium for sound-based interactive experiences. Developing new interfaces for musical expression invariably leads to the appropriation of technology `not meant' or `not for' musical expression, and this is where NIME practice has considerable crossover into the maker / DIY / hardware-hacking subculture. The present thesis approaches ocularcentrism head-on, not through replacing the visual, but through re-ordering, with the aim to even-out, the sensory hierarchy of AR technologies. 

Realistically, however, technologies relating to touch, and especially the chemical senses, present significant barriers compared to the visual and auditory senses in terms of access, cost, expertise, and time. In this research I have attempted to make room for future cases where this technology may be available, by using open-source software and hardware where possible, and by presenting a case for the design of multisensory AR experience in the set of design patterns in \autoref{sec: discussion-patterns}. Thus, the present thesis approaches audiovisual, gestural AR as a starting point for my own AR composition and performance, with a mindful eye (or ear) to the future capabilities of multisensory HCI technologies.

\subsection{AR as Overlay} \label{sec: method-resistance-overlay}
The third form of resistance drawn from in the thesis is that of AR's paradigmatic function as an `overlay' device. So far, the origin of this approach, and contemporary alternatives to it have been explored \autoref{sec: ar-process}, with a case being made by practitioners \citep{mann1994,schraffenberger2018,chevalier2020} towards re-situating AR as a process of perceptual mediation. In the present research, the resistance towards AR functioning as a ``heads-up display'' (HUD) has led to propositions on how these approaches intersect with a `4E's' approach to cognition, later on, in \autoref{sec: discussion-medium-embodiment}. In practice, it has motivated the AR experiences developed in this research to go beyond passively layering sensory stimuli in front of immersants \textit{as experience itself}. This is achieved by presenting experiences that are contingent on immersant movement, gesture, and environmental sounds, not only passive viewership. In this way, the AR experience can be thought more of as a hybridisation or alteration of reality rather than an simple `HUD' augmentation of reality.

As with multisensory technologies though, technologically achieved ``diminished reality'' and other non-augmenting subforms are, in general, more computationally expensive and beyond the scope of the research practice due to lack of expertise and time. By way of acknowledging their importance as valid methods of perceptual mediation, they are accounted for in the design patterns presented in \autoref{sec: discussion-patterns}.



% --------------------------------------------------------------------------- %
\section{Research in Practice}
The three ensuing chapters, \textit{\nameref{sec: area} (2020)}, \textit{\nameref{sec: polaris} (2021)}, and \textit{\nameref{sec: polygons} (2022)} outline and analyse three AR interactive music systems developed over the course of this thesis, each with different but converging focuses. They make up practical contributions to knowledge, as well as forming the grounds for the propositions in \autoref{sec: discussion-medium}. Though they were initially guided by a loose set of proto-design patterns \citep{bilbow2020}, it is through the insights and analysis gained in their creation and iteration that the \nameref{sec: discussion-patterns} in \autoref{sec: discussion} have been developed; they can therefore also be thought of as research probes in their own right.

\textit{area\textasciitilde{}} was the starting point for practical work in the present thesis. It is a non-visual audio AR experience centred around the self-composition of a hybrid listening environment using an ambisonic microphone, head-, and hand-tracking. As will be expanded upon in the next chapter, due to the circumstances of the COVID-19 lockdown, I opted for an autobiographical design method for the iteration and evaluation of the system. Out of the evaluation of this project, namely the endeavour to expand into audiovisual AR experiences, I discovered a suitable platform the develop the rest of doctoral research on: the open-source Project North Star AR headset. \textit{polaris\textasciitilde{}} describes an AR experience that was developed to examine the suitability of the Project North Star AR headset's use in an audiovisual installation context. Participant studies were conducted and an evaluation of their experience was carried out via the grounded theory method of analysis. \textit{polygons\textasciitilde{}} was a direct result of sharing the experience of artistic AR with the participants of \textit{polaris\textasciitilde{}}. In proposing an Sound ARt Performance Practice, it describes the composition of a set of AR musical instruments, technical setup, and considerations of deploying an AR headset as a medium for musical performance.
