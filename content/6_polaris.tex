%%%%%%%%%%%%%%%%%%%%%%%%%%%%%%%%%%%%%%%%%%
\chapter{Evaluating polaris\textasciitilde{}}{An Audiovisual Augmented Reality Experience Build on Open-Source Hardware and Software}

\markboth{}{Evaluating polaris\textasciitilde{} - An Audiovisual Augmented Reality Experience Build on Open-Source Hardware and Software}

\epigraph{\emph{This chapter draws the majority of its content from a published conference paper in New Interfaces for Musical Expression}}{\citep[]{bilbow2021b}}



%%%%%%%%%%%%%%%%%%%%%%%%%%%%%%%%%%%%%%%%%%
\section{Abstract}\label{sec: polaris-abstract}
Augmented reality (AR) is increasingly being envisaged as a process of perceptual mediation or modulation, not only as a system that combines aligned and interactive virtual objects with a real environment. Within artistic practice, this reconceptualisation has led to a medium that emphasises this multisensory integration of virtual processes, leading to expressive, narrative-driven, and thought-provoking AR experiences. This paper outlines the development and evaluation of the polaris~ experience. polaris~ is built using a set of open-source hardware and software components that can be used to create privacy-respecting and cost-effective audiovisual AR experiences. Its wearable component is comprised of the open-source Project North Star AR headset and a pair of bone conduction headphones, providing simultaneous real and virtual visual and auditory elements. These elements are spatially aligned using Unity and PureData to the real space that they appear in and can be gesturally interacted with in a way that fosters artistic and musical expression. In order to evaluate the polaris~, 10 participants were recruited, who spent approximately 30 minutes each in the AR scene and were interviewed about their experience. Using grounded theory, the author extracted coded remarks from the transcriptions of these studies, that were then sorted into the categories of Sentiment, Learning, Adoption, Expression, and Immersion. In evaluating polaris~ it was found that the experience engaged participants fruitfully, with many noting their ability to express themselves audiovisually in creative ways. The experience and the framework the author used to create it is available in a \href{https://github.com/sambilbow/polaris}{Github respository}.


%%%%%%%%%%%%%%%%%%%%%%%%%%%%%%%%%%%%%%%%%%
\section{Augmented Reality in Computational Art}\label{sec: polaris-intro}
In the last twenty years, computational technology has become increasingly expressive, and the systems we engage with have become more interactive. Due to this, the arts (especially forms of sound-driven digital art) have embraced new technologies in tandem with, and often contributing to, their development. This has led to more human-centred and DIY methods of designing tools for digital art creation, often leading towards more performative and experimental interactions. Among the technologies that have seen nascent use in the arts is augmented reality (AR), typically defined as a system that (1) combines real and virtual elements, (2) is interactive in real time, and (3) is registered in 3-D \citep{azuma1997}. Despite a broad set of criteria, the paradigmatic form of AR in recent history has typically been an interface that overlays content onto a participant’s visual field.

In an attempt to break free from such limitations in the context of human-centred design, Mann argues for the term “Mediated Reality” \citep{mann1994}, emphasising that these experiences can offer more than layered content in front of us, rather, that they have the potential to completely mediate our perceived reality. Schraffenberger also sought to address this seemingly sticky archetype of AR simply being “layered information”, by setting out her “Subforms of AR”: extended, diminished, altered, and hybrid reality, and the concept of extended perception \citeyearpar{schraffenberger2018}. Similarly, but in the field of experimental music research, and practice of interactive sound installation, Chevalier and Kiefer recognise AR as “real-time computationally mediated perception” \citeyearpar{chevalier2020}. The emphasis by all four on the mediation of perception, not only resituates AR to include immersive experiences that provide more than just layered content, but also empowers AR applications that engage the non-visual senses, due to the multisensory nature of human perception.

As a digital musician, this reconceptualisation has been instrumental in my viewing AR as a medium for the creation of immersive, multisensory experiences and interfaces for musical expression. Coming to AR experience design through an artistic DIY approach liberates AR from consumer technologies (and their visual biases), which are often prohibitively expensive, require developers’ licenses, or agreement to privacy policies, such as Microsoft’s Hololens 2, Magic Leap’s ML-1, or Facebook / Oculus / Meta’s Quest 2. In my practice, these points of resistance (wanting to push past the paradigms of ocularcentrism \footnote{Defined as the hierarchical elevation of the visual sense over the other senses in Western cultures. \citep{oxfordreference2020}} and layering in AR and engaging with technologies that are cost-effective and privacy-respecting), have led to my finding several hardware and software solutions that have been used in the creation of the polaris~ experience.

The objective of this research is to evaluate polaris~ as an AR experience for its ability to provide a space for gestural audiovisual expression, primarily through a user study, and later using the grounded theory method to extract relevant themes from participant interactions. The outcome of this research will eventually be a set of multisensory AR design guidelines: developed via iteration based on themes that are found in the analysis of participant experiences, and autobiographical design remarks \citep{neustaedter2012}. At present, the experience is available online to download, along with the framework used to create it.



%%%%%%%%%%%%%%%%%%%%%%%%%%%%%%%%%%%%%%%%%%
\section{Design Framework}\label{sec: polaris-framework}
The polaris~ experience itself is built using mostly open-source hardware and software. As well as creating the experience, I was interested in keeping a log of its framework in order to ensure its reproduceability and to facilitate further creation of a wide variety of audiovisual AR experiences. Any ‘artist-developer’ \footnote{By using the term artist-developer, I refer to the media artist, creative coder, digital musician, or indeed any of the other many terms used to describe the category of artist who uses code and/or technology as their medium of artistic expression.} wanting to work on similar experiences should have the ability, when following this framework, to rapidly create and prototype low-cost and privacy-respecting multisensory AR artworks, experiences, and instruments. This section details the framework, but additional information can be found on \href{https://github.com/sambilbow/polaris}{Github} and \href{https://sambilbow.com}{my website}.

\subsection{polaris\textasciitilde{} Hardware}\label{sec: polaris-framework-hardware}
\subsubsection{Project North Star}\label{sec: polaris-framework-hardware-pns}
\subsubsection{Bone Conduction}\label{sec: polaris-framework-hardware-bc}

\subsection{polaris\textasciitilde{} Software}\label{sec: polaris-framework-software}
\subsubsection{Unity}\label{sec: polaris-framework-software-unity}
\subsubsection{PureData}\label{sec: polaris-framework-software-puredata}



%%%%%%%%%%%%%%%%%%%%%%%%%%%%%%%%%%%%%%%%%%
\section{Study Design}\label{sec: polaris-study}
\subsection{Questionnaire}\label{sec: polaris-study-questionnaire}

\subsection{Tutorial}\label{sec: polaris-study-tutorial}

\subsection{The polaris\textasciitilde{} Experience}\label{sec: polaris-study-experience}

\subsection{Interview}\label{sec: polaris-study-interview}



%%%%%%%%%%%%%%%%%%%%%%%%%%%%%%%%%%%%%%%%%%
\section{Participant Feedback}\label{sec: polaris-feedback}
\subsection{Grounded Theory}\label{sec: polaris-feedback-grounded}

\subsection{Sentiment}\label{sec: polaris-feedback-sentiment}

\subsection{Learning}\label{sec: polaris-feedback-learning}

\subsection{Adoption}\label{sec: polaris-feedback-adoption}
\subsubsection{Comfort and Fit}\label{sec: polaris-feedback-adoption-comfort}
\subsubsection{Alignment and Tracking}\label{sec: polaris-feedback-adoption-alignment}
\subsubsection{Uses of AR and comparisons to other media}\label{sec: polaris-feedback-adoption-uses}
\subsubsection{Safety and Accessibility}\label{sec: polaris-feedback-adoption-safety}

\subsection{Expression}\label{sec: polaris-feedback-expression}

\subsection{Immersion}\label{sec: polaris-feedback-immersion}
\subsubsection{Awareness}\label{sec: polaris-feedback-immersion-awareness}
\subsubsection{Sights}\label{sec: polaris-feedback-immersion-sights}
\subsubsection{Sounds}\label{sec: polaris-feedback-immersion-sounds}
\subsubsection{Actions}\label{sec: polaris-feedback-immersion-actions}
\subsubsection{Physicality of Content}\label{sec: polaris-feedback-immersion-physicality}



%%%%%%%%%%%%%%%%%%%%%%%%%%%%%%%%%%%%%%%%%%
\section{Conclusion}\label{sec: polaris-conclusion}
\subsection{Future Work}\label{sec: polaris-conclusion-future}



%%%%%%%%%%%%%%%%%%%%%%%%%%%%%%%%%%%%%%%%%%
\section{Ethics Statement}\label{sec: polaris-ethics}
\subsection{Socio-economic Fairness}\label{sec: polaris-ethics-}
\subsubsection{A note on the environment}\label{sec: polaris-ethics-environment}

\subsection{Study Participants}\label{sec: polaris-ethics-participants}
\subsubsection{Inclusion}\label{sec: polaris-ethics-accessibility}
\subsubsection{Consent}\label{sec: polaris-ethics-inclusion}
\subsubsection{Renumeration}\label{sec: polaris-ethics-renumeration}
\subsubsection{Consent}\label{sec: polaris-ethics-consent}
\subsubsection{Data and privacy}\label{sec: polaris-ethics-data}
