%%%%%%%%%%%%%%%%%%%%%%%%%%%%%%%%%%%%%%%%%%
\chapter{Evaluating polaris\textasciitilde{}}{An Audiovisual Augmented Reality Experience Build on Open-Source Hardware and Software}

\markboth{}{Evaluating polaris\textasciitilde{} - An Audiovisual Augmented Reality Experience Build on Open-Source Hardware and Software}

\epigraph{\emph{This chapter draws the majority of its content from a published conference paper in New Interfaces for Musical Expression}}{\citep[]{bilbow2021b}}



%%%%%%%%%%%%%%%%%%%%%%%%%%%%%%%%%%%%%%%%%%
\section{Abstract}\label{sec: polaris-abstract}
Augmented reality (AR) is increasingly being envisaged as a process of perceptual mediation or modulation, not only as a system that combines aligned and interactive virtual objects with a real environment. Within artistic practice, this reconceptualisation has led to a medium that emphasises this multisensory integration of virtual processes, leading to expressive, narrative-driven, and thought-provoking AR experiences. This paper outlines the development and evaluation of the polaris~ experience. polaris~ is built using a set of open-source hardware and software components that can be used to create privacy-respecting and cost-effective audiovisual AR experiences. Its wearable component is comprised of the open-source Project North Star AR headset and a pair of bone conduction headphones, providing simultaneous real and virtual visual and auditory elements. These elements are spatially aligned using Unity and PureData to the real space that they appear in and can be gesturally interacted with in a way that fosters artistic and musical expression. In order to evaluate the polaris~, 10 participants were recruited, who spent approximately 30 minutes each in the AR scene and were interviewed about their experience. Using grounded theory, the author extracted coded remarks from the transcriptions of these studies, that were then sorted into the categories of Sentiment, Learning, Adoption, Expression, and Immersion. In evaluating polaris~ it was found that the experience engaged participants fruitfully, with many noting their ability to express themselves audiovisually in creative ways. The experience and the framework the author used to create it is available in a \href{https://github.com/sambilbow/polaris}{Github respository}.


%%%%%%%%%%%%%%%%%%%%%%%%%%%%%%%%%%%%%%%%%%
\section{Augmented Reality in Computational Art}\label{sec: polaris-intro}



%%%%%%%%%%%%%%%%%%%%%%%%%%%%%%%%%%%%%%%%%%
\section{Design Framework}\label{sec: polaris-framework}
\subsection{polaris\textasciitilde{} Hardware}\label{sec: polaris-framework-hardware}
\subsubsection{Project North Star}\label{sec: polaris-framework-hardware-pns}
\subsubsection{Bone Conduction}\label{sec: polaris-framework-hardware-bc}

\subsection{polaris\textasciitilde{} Software}\label{sec: polaris-framework-software}
\subsubsection{Unity}\label{sec: polaris-framework-software-unity}
\subsubsection{PureData}\label{sec: polaris-framework-software-puredata}



%%%%%%%%%%%%%%%%%%%%%%%%%%%%%%%%%%%%%%%%%%
\section{Study Design}\label{sec: polaris-study}
\subsection{Questionnaire}\label{sec: polaris-study-questionnaire}

\subsection{Tutorial}\label{sec: polaris-study-tutorial}

\subsection{The polaris\textasciitilde{} Experience}\label{sec: polaris-study-experience}

\subsection{Interview}\label{sec: polaris-study-interview}



%%%%%%%%%%%%%%%%%%%%%%%%%%%%%%%%%%%%%%%%%%
\section{Participant Feedback}\label{sec: polaris-feedback}
\subsection{Grounded Theory}\label{sec: polaris-feedback-grounded}

\subsection{Sentiment}\label{sec: polaris-feedback-sentiment}

\subsection{Learning}\label{sec: polaris-feedback-learning}

\subsection{Adoption}\label{sec: polaris-feedback-adoption}
\subsubsection{Comfort and Fit}\label{sec: polaris-feedback-adoption-comfort}
\subsubsection{Alignment and Tracking}\label{sec: polaris-feedback-adoption-alignment}
\subsubsection{Uses of AR and comparisons to other media}\label{sec: polaris-feedback-adoption-uses}
\subsubsection{Safety and Accessibility}\label{sec: polaris-feedback-adoption-safety}

\subsection{Expression}\label{sec: polaris-feedback-expression}

\subsection{Immersion}\label{sec: polaris-feedback-immersion}
\subsubsection{Awareness}\label{sec: polaris-feedback-immersion-awareness}
\subsubsection{Sights}\label{sec: polaris-feedback-immersion-sights}
\subsubsection{Sounds}\label{sec: polaris-feedback-immersion-sounds}
\subsubsection{Actions}\label{sec: polaris-feedback-immersion-actions}
\subsubsection{Physicality of Content}\label{sec: polaris-feedback-immersion-physicality}



%%%%%%%%%%%%%%%%%%%%%%%%%%%%%%%%%%%%%%%%%%
\section{Conclusion}\label{sec: polaris-conclusion}
\subsection{Future Work}\label{sec: polaris-conclusion-future}



%%%%%%%%%%%%%%%%%%%%%%%%%%%%%%%%%%%%%%%%%%
\section{Ethics Statement}\label{sec: polaris-ethics}
\subsection{Socio-economic Fairness}\label{sec: polaris-ethics-}
\subsubsection{A note on the environment}\label{sec: polaris-ethics-environment}

\subsection{Study Participants}\label{sec: polaris-ethics-participants}
\subsubsection{Inclusion}\label{sec: polaris-ethics-accessibility}
\subsubsection{Consent}\label{sec: polaris-ethics-inclusion}
\subsubsection{Renumeration}\label{sec: polaris-ethics-renumeration}
\subsubsection{Consent}\label{sec: polaris-ethics-consent}
\subsubsection{Data and privacy}\label{sec: polaris-ethics-data}
